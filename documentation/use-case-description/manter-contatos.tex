\subsection{Manter contatos}
\begin{center}
\begin{tabular}{ |p{7cm}|p{7cm}| } 
 \hline
 \textbf {Nome do caso de uso} & Manter contatos \\
 \hline
 \textbf{Descrição geral} & Caso de uso responsável por manter os contatos os quais receberão alertas de emergência. \\
 \hline
 \textbf{Ator Principal} & Usuário comum \\ 
 \hline
 \textbf{Atores Secundários} & Nenhum \\
 \hline
 \textbf{Resumo} & Esse caso de uso tem como finalidade adicionar, excluir, ler ou alterar os contatos salvos pelo usuário. \\
 \hline
 \textbf{Pré-Condições} & O login precisa ter sido realizado \\
 \hline 
 %-----------------------------------FLUXO DE EVENTOS
 \multicolumn{2}{|c|}{\textbf{Fluxo pricipal de eventos} } \\
 \hline
 \textbf{Ações do ator} & \textbf{Ações do sistema} \\
 \hline
 \multicolumn{2}{|c|}{\textbf{Fluxo de cadastro de contatos} } \\
 \hline 
 Selecionar cadastrar contato & Mostrar tela de cadastro \\
 \hline
 Preencher formulário & \\
 \hline
 Clicar no botão “salvar” & Validar entrada de dados \\
 \hline
  & Salvar dados no banco \\
 \hline
 \multicolumn{2}{|c|}{\textbf{Fluxo de deleção de contatos} } \\
 \hline 
 Na tela de listagem de contatos opção excluir contato específico & Mostrar tela de confirmação exclusão \\
 \hline
 Confirmar exclusão & Verificar existência do contato e em seguida excluir \\
 \hline
 \multicolumn{2}{|c|}{\textbf{Fluxo de edição de contatos} } \\
 \hline 
 Na tela de listagem, clicar em alterar contato & Exibir tela de alteração de dados \\
 \hline
 Escolher contato e clicar no botão editar & Exibir tela de edição \\
 \hline
 Inserir dados a serem alterados & \\
 \hline
 Clicar no botão salvar & Validar as informações a serem alteradas\\
 \hline
  & Tela de confirmação de alteração \\
 \hline
 Confirmar no botão "Sim" & Salvar dados no banco \\
 \hline
 \multicolumn{2}{|c|}{\textbf{Fluxo de listagem de contatos} } \\
 \hline 
 Na aba contatos, opção listar contatos & Buscar dados no banco \\
 \hline
 & Exibir dados\\
 \hline
 \textbf{Pós condições} & Sem pós condições \\
 \hline
 \textbf{Observações} & Se houver inserção de dados inválida pelo usuário neste caso de uso, a aplicação irá exibir uma mensagem de erro\\
 \hline
\end{tabular}
\end{center}